\documentclass{article}
\usepackage{tabularx}

\newcolumntype{s}{>{\hsize=0.2\hsize}>{\centering}X}
\newcolumntype{m}{>{\hsize=1.0\hsize}>{\centering}X}
\newcolumntype{b}{>{\hsize=1.8\hsize}>{\centering}X}

% Enter the name of the subject
\newcommand{\assignmentname}{Assignment 4: Testing}
% Your names
\newcommand{\studentA}{Evelien Daems}
\newcommand{\studentB}{Lars Van Roy}

\title{\textmd{\textbf{Software Engineering}}\\\normalsize\vspace{0.1in}\Large{\assignmentname}\\\vspace{0.1in}\small{\textit{3 Ba INF \  2018-2019}}}
\author{\studentA \\ \studentB}

\begin{document}
\maketitle
\underline{Cyclomatic Complexity}\newline
\newline
De cyclomatic complexity kan berekend worden door het aantal edges - het aantal nodes te beschouwen + 2 of het aantal conditie nodes + 1. Beide geven ons in dit geval 10. Deze waarde symboliseert een bovengrens voor het aantal mogelijke onafhankelijke paden door de flowgraph. \\

\maketitle
\underline{independent paths}\newline
\newline
Wanneer we beginnend van het kortst mogelijke pad verdergaan door steeds het volgende korste pad te nemen dat opnieuw onafhankelijk is krijgen we een mogelijkheid om de verschillende paden voor te stellen. Een nieuw pad wordt onafhankelijk genoemd van de voorgaande paden als er in het nieuwe pad een node voorkomt die nog niet voorkwam in de voorgaande paden. Wanneer we dit doen bekomen we de volgende paden. \\

\begin{table}[h]
	\centering
	\begin{tabularx}{\linewidth}{| s | b |}
		\hline
		index & pad \tabularnewline
		\hline
		0 & {1, 2, 6, 7, 8, 9, 10, 7, 21} \tabularnewline
		\hline
		1 & {1, 2, 3, 2, 6, 7, 8, 9, 10, 7, 21} \tabularnewline
		\hline
		2 & {1, 2, 3, 4, 5, 2, 6, 7, 8, 9, 10, 11, 12, 10, 7, 21} \tabularnewline
		\hline
		3 & {1, 2, 3, 4, 5, 2, 6, 7, 8, 9, 10, 11, 12, 13, 14, 15, 16, 17, 18, 19, 20, 12, 10, 12, 10, 7, 10, 12, 10, 7, 21} \tabularnewline
		\hline
	\end{tabularx}
	\caption{alle mogelijke paden}
\end{table}

\newpage
\maketitle
\underline{test cases}\newline
\newline
Bij nadere observatie van deze paden zien we dat het merendeel hiervan niet bereikbaar is. sommige condities kunnen niet falen zonder dat andere in het programma ook falen en omgekeerd. Uiteindelijk blijven volgende 4 paden over met bijhorende input en output. Hierbij zijn triviale inputvelden, of m.a.w. inputvelden waarvoor de waarde niet relevant is om het gewenste pad te bekomen, weggelaten.\\

\begin{table}[h]
	\centering
	\begin{tabularx}{\linewidth}{| s | b | m |}
		\hline
		index & input & output \tabularnewline
		\hline
		0 & m\_db\.m\_tidlist = \{\} & \{\{\}, \{\}\} \tabularnewline
		\hline
		1 & m\_db\.m\_tidlist = \{0:\{0\}\}; \newline min\_sup = 2; & \{\{\}, \{\}\} \tabularnewline
		\hline
		2 & m\_db\.m\_tidlist = \{0:\{0\}\}; \newline min\_sup = 1; & \{\{[0]\}, \{\}\} \tabularnewline
		\hline
		3 & m\_db\.m\_tidlist = \{0:\{1,2\}, 1:\{1\}\}; \newline min\_sup = 1; & \{\{[0], [1]\}, \{[0, 1]\}, \{\}\} \tabularnewline
		\hline
	\end{tabularx}
	\caption{test cases}
\end{table}

\end{document}