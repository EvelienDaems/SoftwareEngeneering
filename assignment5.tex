\documentclass{article}
\usepackage{tabularx}
\usepackage{pdflscape}
\usepackage{graphicx}

\graphicspath{ {./} }

\newcolumntype{s}{>{\hsize=0.2\hsize}>{\centering}X}
\newcolumntype{m}{>{\hsize=1.0\hsize}>{\centering}X}
\newcolumntype{b}{>{\hsize=1.8\hsize}>{\centering}X}

% Enter the name of the subject
\newcommand{\assignmentname}{Assignment 4: Testing}
% Your names
\newcommand{\studentA}{Evelien Daems}
\newcommand{\studentB}{Lars Van Roy}

\title{\textmd{\textbf{Software Engineering}}\\\normalsize\vspace{0.1in}\Large{\assignmentname}\\\vspace{0.1in}\small{\textit{3 Ba INF \  2018-2019}}}
\author{\studentA \\ \studentB}

\begin{document}
\maketitle
\noindent
\textbf{Calculate() function analysis} \newline
\newline
\maketitle
\noindent
\underline{Control Flow Graph}\newline
\newline
Een control flow graph is een visuele weergave van een programma waarmee alle mogelijke paden doorheen een programma kunnen gegenereerd worden. Elke binaire node (een node waar twee pijlen uitkomen) komt overheen met een conditie in het programma. Elke mogelijke sequentie van nodes dat eindigt in een eindstaat is een mogelijk pad door het programma, het is echter niet gegarandeerd dat dit pad ook bestaat. Sommige condities kunnen enkel slagen indien andere condities ook slagen en of slagen sowieso x aantal keren. Het is dus niet omdat de sequentie bestaat in de control flow graph dat het ook een mogelijk pad is. Voor de flow graph van de calculate() functie zie appendix A. \\
\newline
\maketitle
\noindent
\underline{Cyclomatic Complexity}\newline
\newline
De cyclomatic complexity kan berekend worden door het aantal edges - het aantal nodes te beschouwen + 2 of het aantal binaire conditie nodes + 1. Beide geven ons in dit geval 10. Deze waarde symboliseert een bovengrens voor het aantal mogelijke onafhankelijke paden door de flowgraph. \\
aantal edges - aantal nodes + 2 = 29 - 21 + 2 = 10 \\
aantal binarie condities + 1 = 9 + 1 = 10 \\

\newpage
\maketitle
\noindent
\underline{independent paths}\newline
\newline
Wanneer we beginnend van het kortst mogelijke pad verdergaan door steeds een nieuw onafhankelijke pad te nemen krijgen we een mogelijkheid om de verschillende paden voor te stellen. Een nieuw pad wordt onafhankelijk genoemd van de voorgaande paden als er in het nieuwe pad een node voorkomt die nog niet voorkwam in de voorgaande paden. Wanneer we dit doen bekomen we de volgende paden. \\

\begin{table}[h]
	\centering
	\begin{tabularx}{\linewidth}{| s | b |}
		\hline
		index & pad \tabularnewline
		\hline
		0 & {1, 2, 6, 7, 21} \tabularnewline
		\hline
		1 & {1, 2, 6, 7, 8, 9, 10, 7, 21} \tabularnewline
		\hline
		2 & {1, 2, 3, 2, 6, 7, 8, 9, 10, 7, 21} \tabularnewline
		\hline
		3 & {1, 2, 3, 4, 5, 2, 6, 7, 21} \tabularnewline
		\hline
		4 & {1, 2, 3, 4, 5, 2, 6, 7, 8, 9, 10, 11, 12, 10, 7, 21} \tabularnewline
		\hline
		5 & {1, 2, 3, 4, 5, 2, 6, 7, 8, 9, 10, 11, 12, 13, 14, 12, 10, 7, 21} \tabularnewline
		\hline
		6 & {1, 2, 3, 4, 5, 2, 6, 7, 8, 9, 10, 11, 12, 13, 14, 15, 12, 10, 7, 21} \tabularnewline
		\hline
		7 & {1, 2, 3, 4, 5, 2, 6, 7, 8, 9, 10, 11, 12, 13, 14, 15, 16, 17, 12, 10, 7, 21} \tabularnewline
		\hline
		8 & {1, 2, 3, 4, 5, 2, 6, 7, 8, 9, 10, 11, 12, 13, 14, 15, 16, 17, 18, 12, 10, 7, 21} \tabularnewline
		\hline
		9 & {1, 2, 3, 4, 5, 2, 6, 7, 8, 9, 10, 11, 12, 13, 14, 15, 16, 17, 18, 19, 20, 12, 10, 12, 10, 7, 10, 12, 10, 7, 21} \tabularnewline
		\hline
	\end{tabularx}
	\caption{alle mogelijke paden}
\end{table}

\newpage
\maketitle
\noindent
\underline{test cases}\newline
\newline
Bij nadere observatie van deze paden zien we dat het merendeel hiervan niet bereikbaar is. sommige condities kunnen niet falen zonder dat andere in het programma ook falen en omgekeerd. Uiteindelijk blijven volgende 4 paden over met bijhorende input en output. Hierbij zijn triviale inputvelden, of m.a.w. inputvelden waarvoor de waarde niet relevant is om het gewenste pad te bekomen, weggelaten. Een volledig overzicht van alle paden en welke paden nu uiteindelijk een bestaand pad door de functie hadden zijn weergegeven in appendix B.\\

\begin{table}[h]
	\centering
	\begin{tabularx}{\linewidth}{| s | b | m |}
		\hline
		index & input & output \tabularnewline
		\hline
		1 & m\_db\.m\_tidlist = \{\} & \{\{\}, \{\}\} \tabularnewline
		\hline
		2 & m\_db\.m\_tidlist = \{0:\{0\}\}; \newline min\_sup = 2; & \{\{\}, \{\}\} \tabularnewline
		\hline
		4 & m\_db\.m\_tidlist = \{0:\{0\}\}; \newline min\_sup = 1; & \{\{[0]\}, \{\}\} \tabularnewline
		\hline
		9 & m\_db\.m\_tidlist = \{0:\{1,2\}, 1:\{1\}\}; \newline min\_sup = 1; & \{\{[0], [1]\}, \{[0, 1]\}, \{\}\} \tabularnewline
		\hline
	\end{tabularx}
	\caption{test cases}
\end{table}

\newpage
\maketitle
\noindent
\textbf{JUnit testing}\newline
\newline

\maketitle
\noindent
\underline{unit tests}\newline
\newline
Unit tests zijn testen waarbij kleine stukjes code apart worden getest zonder afhankelijkheden etc. Door dit te doen kunnen we specifiek zoeken naar fouten in de implementatie binnen een unit. Voor elke unit worden er meerdere testcases doorlopen. Voor het core.Clients packet onderscheiden we 6 verschillende units, zijnde: Address, Client, Mailing\_Preferences, RegisteredClient, UnregisteredClient en Clients. \newline
\newline
Voor elk van deze units schrijven we voor elke methode één of meerdere testen waarbij we de verwachte output vergelijken met de eigenlijke output.\newline
\newline
Dit lukt voor 3 van de 6 units maar bij de 3 andere zijn er assertions die foutieve returns opvangen. \newline
\newline
In de unit Client faalt de methode getRegisteredClient(string, string) om de gewenste return te geven. Deze zou, indien er bestaande data wordt opgegeven, een referentie moeten geven naar het gerefereerde Client object maar dit doet hij niet, hij returned null onafhankelijk van de gegeven input.\newline
\newline
In de unit Mailing\_preferences faalt de functie third\_party() in sommige gevallen. Wanneer de constructor input gelijk is aan (true, true, true) is er geen probleem maar wanneer de constructor input gelijk is aan (true, true, false) geeft de funtie third\_party() plots ook true terug. Wanneer de constructor input krijgt gelijk aan (false, false, false) werkt alles wel weer normaal. \newline
\newline
Ten slotte heeft de unit RegisteredClient nog een probleem in de functie checkPassword(string). Deze functie zou true moeten return indien het correcte wachtwoord wordt meegegeven als string maar dit is niet het geval.

\newpage
\maketitle
\noindent
\underline{integration tests}\newline
\newline
Integration tests zijn testen waarbij meerdere units samen getest worden. Door dit te doen kunnen we zoeken naar fouten in afhankelijkheden in units. Indien alle unittesten slagen weten we dat er niet zozeer problemen zijn in de units op zich maar dat er fouten zijn in het gebruik van de methoden van andere units.\newline
\newline
Deze test bestaat in essentie uit hetzelfde als de unit tests. We maken opnieuw gebruik van verschillende methodes maar in plaats van enkel methodes van 1 unit te beschouwen beschouwen we meerdere methoden uit meerdere units en het effect dat ze op andere methoden kunnen hebben. \newline
\newline
In de beschouwde integration test worden de calls naar methoden uit core.Clients in het pakket Core onderzocht. Deze methoden worden voor zover de test reikt goed gebruikt en de teruggegeven waarden komen steeds overeen met de verwachte output.\newline
\newline
\maketitle
\noindent
\underline{synopsis voor testen}\newline
\newline
Testen zijn nooit volledig, er zijn steeds methoden die voor sommige situaties wel slagen en voor andere niet. Het is zeer moeilijk om af te wegen wanneer er genoeg testen geschreven zijn. De bedoeling is niet dat het uren duurt om deze testen uit te voeren voor methodes die geen vitale belangen hebben in het verloop van het programma en al evenmin dat er maar een paar testen worden geschreven voor functionaliteiten waarop het hele programma gebaseerd is. Het is dus belangrijk om hier een afweging in te maken.\newline
\newline
Ook al zijn niet altijd alle situaties beschouwd, door een percentage van alle mogelijke situaties te beschouwen en hierop extensief te testen beperken we wel het risico op fouten. Er is namelijk een deel van de situaties die zich kunnen voordoen die getest zijn en waarin niets meer zou mogen vastlopen. \newline
\newline
Ten slotte is het ook belangrijk om zowel unit tests als integration tests te gebruiken. Unit tests zijn zeer handig om fouten binnen units op te zoeken maar deze zijn niet toereikend over het gehele systeem. Fouten die ontstaan door combinaties van units zijn veel voorkomend en moeilijker te vinden dan fouten in units zelf dus is het ook belangrijk om integration tests te schrijven.

\newpage
\maketitle
\noindent
\textbf{appendix A}\newline
\newline
\includegraphics[width=\textwidth]{control_flow_graph}

\newpage
\begin{landscape}
	\maketitle
	\noindent
	\textbf{appendix B}\newline
	\newline
	\begin{table}[h]
		\centering
		\begin{tabularx}{\linewidth}{| >{\hsize=0.3\hsize}>{\centering}X | >{\hsize=1.35\hsize}>{\centering}X | >{\hsize=1.35\hsize}>{\centering}X | >{\hsize=1\hsize}>{\centering}X |}
			\hline
			index & pad & input & output \tabularnewline
			\hline
			0 & {1, 2, 6, 7, 21} & bestaat niet & bestaat niet\tabularnewline
			\hline
			1 & {1, 2, 6, 7, 8, 9, 10, 7, 21} & m\_db\.m\_tidlist = \{\} & \{\{\}, \{\}\} \tabularnewline
			\hline
			2 & {1, 2, 3, 2, 6, 7, 8, 9, 10, 7, 21} & m\_db\.m\_tidlist = \{0:\{0\}\}; \newline min\_sup = 2; & \{\{[0]\}, \{\}\} \tabularnewline
			\hline
			3 & {1, 2, 3, 4, 5, 2, 6, 7, 21} & bestaat niet & bestaat niet \tabularnewline
			\hline
			4 & {1, 2, 3, 4, 5, 2, 6, 7, 8, 9, 10, 11, 12, 10, 7, 21} & m\_db\.m\_tidlist = \{0:\{0\}\}; \newline min\_sup = 1; & \{\{[0]\}, \{\}\} \tabularnewline
			\hline
			5 & {1, 2, 3, 4, 5, 2, 6, 7, 8, 9, 10, 11, 12, 13, 14, 12, 10, 7, 21} & bestaat niet & bestaat niet\tabularnewline
			\hline
			6 & {1, 2, 3, 4, 5, 2, 6, 7, 8, 9, 10, 11, 12, 13, 14, 15, 12, 10, 7, 21} & bestaat niet & bestaat niet \tabularnewline
			\hline
			7 & {1, 2, 3, 4, 5, 2, 6, 7, 8, 9, 10, 11, 12, 13, 14, 15, 16, 17, 12, 10, 7, 21} & bestaat niet & bestaat niet \tabularnewline
			\hline
			8 & {1, 2, 3, 4, 5, 2, 6, 7, 8, 9, 10, 11, 12, 13, 14, 15, 16, 17, 18, 12, 10, 7, 21} & bestaat niet & bestaat niet \tabularnewline
			\hline
			9 & {1, 2, 3, 4, 5, 2, 6, 7, 8, 9, 10, 11, 12, 13, 14, 15, 16, 17, 18, 19, 20, 12, 10, 12, 10, 7, 10, 12, 10, 7, 21} & bestaat niet & bestaat niet \tabularnewline
			\hline
		\end{tabularx}
		\caption{paden + test cases}
	\end{table}
\end{landscape}

\end{document}